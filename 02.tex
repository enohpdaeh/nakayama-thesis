\chapter{研究内容}
\label{chap:contents}

本章では、本研究の内容を説明する。
\section{概要}

\section{システム概要}

本システムはの目的はあらゆる情報を容易に取得することだ。
サーバーは各種APIなどを用いてあらゆる情報を取得しLindaへと書き込み、Androidクライアントからのリクエストに応じてLindaからデータを取得しJSON形式でレスポンスを返す。
Androidクライアントはユーザーの操作に応じてサーバーにリクエストを送り、返されたJSONをパースして表示する。

以下にシステムの構成を示す。大枠として、Linda\footnote{http://linda.masuilab.org\\データをクラウド上で共有するためのフレームワーク。並列処理で同時に多くのクライアントを処理できる。}からサーバーがデータを取得し、クライアントのリクエストに応じて送信する。

\begin{table}[htbp]
  \caption{システム構成}
  \label{tb:files}
  \begin{center}\begin{tabular}{c|l}
    \hline
    システム&概要\\\hline\hline
    {\tt Linda}&データ元。データストリームからデータを取得する。\\\hline
    {\tt サーバー}&Lindaからデータを取得し、クライアントからのリクエストが来たら送信する。\\
                      &Lindaにデータを書き込む。\\\hline
    {\tt WEBクライアント}&取得する情報を選択する為のビューを提供する。\\\hline
    {\tt Androidクライアント}&サーバーにデータを要求し、ユーザーに情報を表示する。\\\hline
    {\tt データベース}&サーバーのデータを保存している。\\\hline
  \end{tabular}\end{center}
\end{table}

\subsection{サーバーサイドのシステム構成}

\subsection{Androidサイドのシステム構成}

\nocite{*}
