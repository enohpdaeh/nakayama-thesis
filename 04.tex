\chapter{考察}
\label{chap:consideration}

\section{本研究についての考察}
A 「結果」(調査や分析の結果)を解釈して「目的」で述べられた研究課題への回答となる諸命題を引き出す。
(結論として何が分かったのかを示す。このAが「考察」の中核部分である。)

B かかる命題の成立条件や限定範囲があれば但し書きをつける。(命題とその付帯条件が結論となる)

C 「対象と方法」で記述した、調査や分析の手続きについて弱点ないし誤解を生じやすい点があれば、補足説明をする。

D 自ら提起する新たな知見と先行研究との関係に言及する。

現在、インターネット上には様々な情報が溢れている。それだけでなく個人が自ら情報を発信することも容易になっている。しかし、様々な情報を一括で取得する手法は確立されていない。

テキストデータしかとれないが、取りたい情報がテキストデータとは限らない。
見やすさを考えた時に、現状がベストとは言えない。
自分で使ってどう感じたのかを書く。(成功だったのかどうか)
