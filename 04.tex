\chapter{考察}
\label{chap:consideration}

\section{本研究についての考察}
A 「結果」(調査や分析の結果)を解釈して「目的」で述べられた研究課題への回答となる諸命題を引き出す。
(結論として何が分かったのかを示す。このAが「考察」の中核部分である。)

B かかる命題の成立条件や限定範囲があれば但し書きをつける。(命題とその付帯条件が結論となる)

C 「対象と方法」で記述した、調査や分析の手続きについて弱点ないし誤解を生じやすい点があれば、補足説明をする。

D 自ら提起する新たな知見と先行研究との関係に言及する。
問題は解決できたのか。
自分で使ってどう感じたのかを書く。(成功だったのかどうか)

現在、インターネット上には様々な情報が溢れている。それだけでなく個人が自ら情報を発信することも容易になっている。

作成したシステムを実際に使用して気づいた点
問題の解決に成功した点
不要な情報が表示されない
情報の取捨選択が容易

問題点
- テキストデータしかとれないが、取りたい情報がテキストデータとは限らない。
- 見やすさを考えた時に、現状がベストとは言えない。
