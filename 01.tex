\chapter{序論}
\label{chap:introduction}
\section{背景}
なぜこの研究が必要だったのか、しっかりした理屈を考えておく。
現状どういう問題があるのか
どういう方法で解決を試みたのか
結果として上手く行ったのか、いかなかったのか

 情報を獲得する手法としてAndroid Widgetは普遍的に用いられている。
天気予報を見たければ天気予報のWidgetを用いる必要があり、別の情報が必要であればまた別のWidgetを用いる必要がある。
しかし、Android端末の一つの画面に置くことの出来るWidgetの数は限られているため獲得できる情報の種類も限られてしまう。
また、自分の取得したい情報に対応したWidgetが存在しない場合、自らWidgetを作成するしか無かった。
 そこで、あらゆる情報を容易にWidgetとして表示することの出来る汎用的なWidgetを作成することで解決を図った。

\section{目的}
 あらゆる情報を容易に取得する。

\section{本文書の構成}
第\ref{chap:introduction}章では本研究の概要を書いた。

第\ref{chap:contents}章では研究内容を説明する。第\ref{chap:prototype}章ではプロトタイプの実装方法を解説する。第\ref{chap:consideration}章では考察を書く。最後に第\ref{chap:conclusion}章にて結論を書き本論文をしめることとする。添付として参考文献を追記する。
